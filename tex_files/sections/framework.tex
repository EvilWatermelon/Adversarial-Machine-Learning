\section{The conceptual framework}
\label{sec:conFrame}

In contrast to Schwerdtner et al., the framework of this thesis concentrates on training, especially Risk Measurement before and during training of the ML model.
The conceptual framework discusses and explains the RMF. The RMF is a conceptual and technical framework which measures risks of backdoor attacks and measures the attacker effort. The attacker effort
is measured by objective properties. These objective properties are the base of the risk indicators for the attacker effort explained in the following subsection. Objective properties

\subsection{UML diagrams}


\subsection{Finding the attacker's effort}

\subsubsection*{Using threat models to find risk indicators to measure the attackers effort}

\subsection{Characteristics of backdoor attacks}

\subsection{Risk indicators}
\label{sec:risk_indicators}

The RMF measure risks by so called risk indicators. Properties, attributes and proposals are the basis for the risk indicators, among other things. Breier et al. in subsection \ref{sec:approaches} present proposals that are the approach for the proposals for the risk indicators.

\newpage

\section{The technical framework}
\label{sec:techFrame}

The technical RMF uses Python 3.7 as the programming language and ART as the basis. Beside the attacks given by the ART, there is a function from the technical RMF to execute individual attacks. This technical RMF should be used a step ahead of using the framework of Schwerdtner et al.

\subsection{Using ART as the basis for the technical framework}

\subsection{Implementing additional attacks}

Beside the attacks that are called from functions of ART it must be possible to implement and execute new attacks for the evaluation to measure the attackers knowledge, skills and extent of damage.

\subsection{Implementation of the logging function}

Show measured risks is able with logging from the Python logging module. The function waits for two parameters. A message string and the wanted logging level (i.e. INFO or DEBUG). The called log function in the RMF could look like this:
\begin{lstlisting}
  log(f"{variable_name}", 'INFO')
\end{lstlisting}

\subsection{Implementation of the visualization}

\subsection{Build in the risk indicatiors}
