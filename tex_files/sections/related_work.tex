\section{Related Work}
\label{sec:relWork}

This chapter presents the relevant background knowledge and show approaches from other scientific paper.

\subsection{ISO 27004}

This present thesis use the requirements of the ISO 27004. ISO 27004 is a security standard from the ISO \cite{DBLP:conf/euspn/MeriahR19} 27000 family  which guides on continious basis evaluation methods. The present ISO can be related with ISO 27001 or used as a standalone standard. The ISO 27004 standard specifies what to be measured, when the measurement is needed and types of measurement \cite{lundholm2011design}. The effectivness measurement is the type of measurement for the SMF.

\subsection{Security risks in context of Machine Learning}

Xiao et. al \cite{DBLP:conf/sp/XiaoLZX18} evaluate the security risks in deep
learning for common frameworks i.e. TensorFlow. Xiao et. al uses the framework
sample applications along the frameworks. One statement of Xiao et. al is that the
named frameworks TensorFlow, Caffe and Torch are implemented with many lines of code
which make them vulnerable for many security vulnerabilities i.e. heap overflow or
integer overflow.

\subsection{Risk assessment in context of Machine Learning}

Paul Schwerdtner et. al \cite{DBLP:journals/corr/abs-2011-04328} present in their
work a framework to evaluate ML model by input corrupted data. This thesis discuss
this paper as an approach to estimate where the SMF could be used for.

\subsection{Adversarial-Robustness-Toolbox}

For this present thesis the technical framework Adversarial-Robustness-Toolbox (ART)
\cite{art2018} is a main component.

\subsection{Approaches}

Jakub Breier et. al \cite{DBLP:journals/corr/abs-2012-04884} propose in their paper
different proposals to measure risks with different aspects. \\
