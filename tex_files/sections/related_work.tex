\tikzset{
    leftNode/.style={circle,minimum width=.5ex, fill=none,draw},
    rightNode/.style={circle,minimum width=.5ex, fill=black,thick,draw},
    rightNodeInLine/.style={solid,circle,minimum width=.7ex, fill=black,thick,draw=white},
    leftNodeInLine/.style={solid,circle,minimum width=.7ex, fill=none,thick,draw},
  }

\section{Related Work}
\label{sec:relWork}

This chapter presents the relevant background knowledge and show approaches from other scientific paper.

\subsection{ISO/IEC 27004:2009}

This present thesis based the requirements of Risk measurement of ISO 27004, among other things. ISO 27004 is a international security standard from the ISO 27000 \cite{DBLP:conf/euspn/MeriahR19} family which guides on continious basis evaluation methods. The present ISO can be related with ISO 27001 or used as a standalone standard. In ISO 27001 it is declared as a requirement where the effectiveness must be measured of a Information Security Management System \cite{barabanov2011information}. The ISO 27004 standard specifies what to be measured, when the measurement is needed and types of measurement \cite{lundholm2011design}. Barabanov et al. \cite{barabanov2011information} and Tarnes \cite{tarnes2012information} describe in their works the different properties of ISO/IEC 27004:2009 for Risk measurement. Tarnes shows the information security measurement model which is shown in Figure \ref{fig:is_measurement_model}.

\begin{figure}[h!]
  \centering
  \includegraphics[width=8cm]{pictures/is_measurement_model.jpg}
  \caption{The information security measurement model \cite{tarnes2012information}}
  \label{fig:is_measurement_model}
\end{figure}

For this thesis the objects to be measured and the measurement are the important parts of the information security measurement model. The measurement method is the SMF which measure based on different properties that are derived from risk indicators that will be discussed in Subsection \ref{sec:risk_indicators}. The attributes in Figure \ref{fig:is_measurement_model} are the properties in the SMF.

\subsection{Approaches from Jakub Breier et. al and Paul Schwerdtner et. al}

This present thesis is divided into two approaches. Jakub Breier et al. \cite{DBLP:journals/corr/abs-2012-04884} propose in their paper different proposals to measure risks with different aspects.
These attacks are used in this thesis as properties to classify attacks. These different properties are attack specificity, attack time and attacker's knowledge. Attack time is split in training time
and deployment time. Training time is the attack time when the model gets manipulated while it trains. Deployment time is the attack time when the hacker attacks a ML model after its release.
Attacker's knowledge is the amount of information the hacker has available. Attackers specificity is the amount an attacker needs to manipulate the output of a ML model. These three properties may
serve as a basis for further properties useful for risk measurement. \\
Paul Schwerdtner et al. \cite{DBLP:journals/corr/abs-2011-04328} is the second approach of this thesis. Schwerdtner et al. show a technical framework to evaluate the risks for ML models. Schwerdtner et
al. give an evaluation whether it is secure to deploy a ML model or not. The ML model in Schwerdtner et al. must be a fully developed ML model that is trained and tested. Schwerdtner et al. concentrate
on inference data when the ML model is executed. This thesis discuss this paper as an approach to estimate where the RMF could be used for.

\subsection{Security risks in context of Machine Learning}

Xiao et al. \cite{DBLP:conf/sp/XiaoLZX18} evaluate the security risks in deep learning for common frameworks for example TensorFlow. Xiao et al. uses the framework sample applications along the
frameworks. One statement of Xiao et. al is that the named frameworks TensorFlow, Caffe and Torch are implemented with many lines of code which make them vulnerable for many security vulnerabilities
for example heap overflow or integer overflow. Xiao et. al work is only in context of deep learning e.g. only for neural networks.

\subsection{Risk assessment in context of Machine Learning}

In addition to ISO 27004, the paper by Sendi et al. \cite{DBLP:journals/compsec/SendiAC16} shows at which point in IT security management risk measurement takes place for the thesis and how it is carried out. In their paper, Sendi et al. evaluated 125 works published between 1995 and 2014. They developed categories and the last category is risk measurement. This category is the last step of risk assessment. To evaluate risks by measuring them, there are different properties which have an impact for risk measurement. Sendi et al. explain that the type of the attack, the dependency severity between resources and the type of defined permissions between resources are needed to measure risks.

\subsection{Adversarial-Robustness-Toolbox}

For this thesis the technical framework Adversarial-Robustness-Toolbox (ART) \cite{art2018} is a main component. Nicolae et al. \cite{DBLP:journals/corr/abs-1807-01069} evaluate in their work the
technical framework ART. ART is a Python library that supports several ML frameworks for example TensorFlow and PyTorch to increase the defense of ML models. ART support 39 attacks and 29 defense'
functions. This thesis only focuses on the attack functions for poisoning attacks which will be discussed in the following section more detailed. The backdoor attacks in the technical framework
ART are introduced by Gu et al. \cite{DBLP:journals/corr/abs-1708-06733}.

\subsubsection*{Backdoor Attacks}
\label{sec:backdoor}
Due to the rising amount of training data, human supervision to check trustworthiness is less possible. That exposes vulnerabilites in training datasets like backdoors. Backdoor attacks can cause far-
reaching consequences for example bypass critical authentication. In \cite{DBLP:journals/corr/abs-2003-03675} Salem et al. introduces dynamic backdoors to trigger (a secret pattern of neighboring
pixels) random patterns and locations to reduce the efficacy on identifying backdoors. Salem et al. discuss in their work three backdoors, Random Backdoor, Backdoor Generating Network and Conditional
Backdoor Generating Network.
Gu et al. show in their paper different backdoor attacks and do a case study with a traffic sign detection attack. The evaluated backdoors are a single pixel backdoor and a pattern backdoor. The single
pixel backdoor changes a pixel to a bright pixel and the pattern backdoor adds a pattern of bright pixels in an image. The implemented attacks from Gu et al. are single target attack and an all-to-all attack.

\subsection{Support-Vector-Machine}

Support-Vector-Machine (SVM) is a supervised ML algorithm which classifies a set of objects (splitted in two groups) between a hyperplane in an \textit{N-dimensional} coordinate system. The goal is to find the maximum distance between the objects in both classes. As the name SVM says, this ML algorithm uses Support Vectors. That are the objects close to the hyperplane. The most maximized margin between the sets of objects is the best hyperplane. When the set of objects are more complex the SVM needs a higher dimensional hyperplane. The following example shows a two dimensional hyperplane. If linear separation is not possible a so called kernel realizes the non-linear to a feature space. \\ \\

\begin{adjustbox}{center}
  \begin{tikzpicture}[
          scale=2,
          important line/.style={thick}, dashed line/.style={dashed, thin},
          every node/.style={color=black},
      ]
      \centering
      \draw[dashed line, yshift=.7cm]
         (.2,.2) coordinate (sls) -- (2.5,2.5) coordinate (sle)
         node[solid,circle,minimum width=2.8ex,fill=none,thick,draw] (name) at (2,2){}
         node[leftNodeInLine] (name) at (2,2){}
         node[solid,circle,minimum width=2.8ex,fill=none,thick,draw] (name) at (1.5,1.5){}
         node[leftNodeInLine] (name) at (1.5,1.5){}
         node [above right] {$w\cdot x + b > 1$};

      \draw[important line]
         (.7,.7) coordinate (lines) -- (3,3) coordinate (linee)
         node [above right] {$w\cdot x + b = 0$};

      \draw[dashed line, xshift=.7cm]
         (.2,.2) coordinate (ils) -- (2.5,2.5) coordinate (ile)
         node[solid,circle,minimum width=2.8ex,fill=none,thick,draw] (name) at (1.8,1.8){}
         node[rightNodeInLine] (name) at (1.8,1.8){}
         node [above right] {$w\cdot x + b < -1$};

      \draw[very thick,<->] ($(sls)+(.2,.2)$) -- ($(ils)+(.2,.2)$)
         node[sloped,above, near end] {Margin};

      \foreach \Point in {(.9,2.4), (1.3,2.5), (1.3,2.1), (2,3), (1,2.9)}{
        \draw \Point node[leftNode]{};
      }

      \foreach \Point in {(2.9,1.4), (2.3,.5), (3.3,.1), (2,0.9), (2.5,1)}{
        \draw \Point node[rightNode]{};
      }
    \end{tikzpicture}
\end{adjustbox}


\subsubsection*{Hyperplane}

The hyperplane is in a SVM a linear line between a set of objects (one set of object is called a class on one side of a hyperplane). The line differentiate the set of objects for classification. The hyperplane is used for two-dimensional coordinate systems.

\subsubsection*{Support Vector}

Support Vectors are the minimum margin on both sides of the hyperplane. The maximum margin is the nearest object to the hyperplane in both classes.

\subsubsection*{SVM optimization}

\subsubsection*{The kernel trick}

The kernel trick is used if the positions of the sets of objects is not redundant to classify them with a hyperplane. Kernel trick is also used if there are more than two classes to classify. If there
are more than two classes the SVM do a multi-class classification. The idea of multi-class classification is separating the classes in a binary classification \cite{tzotsos2008support}.
