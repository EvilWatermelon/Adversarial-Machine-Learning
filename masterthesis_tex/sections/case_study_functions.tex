% !TEX root = C:\Users\Jan\Documents\dev\Risk-Measurement-Framework\masterthesis_tex\masterthesis_main.tex
\section{Case Study functions}
\label{sec:case_study_functions}

\textbf{class\_num}
This argument gets the key value from the labels which is an integer. \\ \\
\textbf{train\_number}
This argument gets the number of images from a labeled folder. \\ \\
\textbf{train\_path}
This argument gets the path where the local training data is stored. \\ \\
\textbf{data\_dir}
This argument gets the path where all local images are stored. \\ \\
\textbf{image\_data}
This argument gets an array with all images from a label. \\ \\
\textbf{image\_labels}
This argument gets the label which belongs to the corresponding images. \\ \\
\textbf{X\_train}
This argument gets the training data. \\ \\
\textbf{attack}
Name of a backdoor attack as a string. \\

\begin{lstlisting}
  createModel(X_train)
\end{lstlisting}

\noindent This function creates the Keras ML model.

\begin{lstlisting}
  read_training_data(train_path, data_dir)
\end{lstlisting}

\noindent This function reads in the training data, calls the \textit{dataset\_visualization()} function and resize the images to $30x30$ pixels. The function is called by the \textit{preprocessing()} function.

\begin{lstlisting}
  preprocessing(train_path, data_dir, image_data, image_labels)
\end{lstlisting}

\noindent After calling the \textit{read\_training\_data()} function, this function assign the \textbf{image\_date} and \textbf{image\_labels} arguments to shuffle the training data. It also can takes a number of images that is lower than the original to execute the ML model faster for test purposes.

\begin{lstlisting}
  model_training(train_path, data_dir, image_data, image_labels)
\end{lstlisting}

\noindent After calling the \textit{preprocessing()} function, this function calls the \textit{AdversarialTrainerMadryPGD()} class which is an ART class to use the \textit{CleanLabelBackdoor} attack. This class must be used because the attacks only takes these classes for the proxy model training.

\begin{lstlisting}
  read_test_data(train_path, data_dir, image_data, image_labels)
\end{lstlisting}

\noindent After calling the \textit{model\_training()} function, this function reads in the test data, resizes the images to $30x30$ pixels and then reshape the images. The last step is the prediction with the test data. This function also calls all RMF functions to evaluate the risk indicators.
