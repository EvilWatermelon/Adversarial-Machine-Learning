% !TEX root = C:\Users\Jan\Documents\dev\Risk-Measurement-Framework\masterthesis_tex\masterthesis_main.tex
\section{Conclusion}
\label{sec:conclusion}

This section summarizes this thesis, discusses the results related to the goals, research questions, and hypotheses from section \ref{sec:intro}. Furthermore, this section explain a possible future work based on the concept, implementation, and evaluation of the current state of the RMF.

\subsection{Discussion of the results}

The ISO 27004 standard is very generalized. That makes it possible to design and implement frameworks which are not covered as a standard or intended to be. The attack time is a risk indicators that measures two different outputs. This could be splitted up into two different risk indicators which would decrease the need to map it.

\subsection{Future work}

After implementing only backdoor attacks it should be possible in the future to use attacks for different attack times. That could increase the knowledge for vulnerabilites in the measured ML model. Further, the RMF could show how to prevent those attacks. This risk matrix is a component for which the appropriate incorporation of the risk value is still missing. Jianxing et al. \cite{DBLP:journals/access/JianxingHSH21} explain in their work that the level of risk could be suitably integrated on knowledge-based rules which this thesis did not do. This is because there is no knowledge-base which this thesis could use to integrate the risk value.
