% !TEX root = C:\Users\Jan\Documents\dev\Risk-Measurement-Framework\masterthesis_tex\masterthesis_main.tex
\section{Conclusion}
\label{sec:conclusion}

This work shows a concept and implementation of the technical framework RMF to measure risks of an attack on a ML model and also measures the attacker's effort to successfully execute this attack. Since this field has a great potential to improve ML security, there are still problems to depict the attacker's effort. \\
This section summarizes this thesis, discusses the results related to the goals, research questions, and hypotheses from section \ref{sec:intro}. Furthermore, this section explains a possible future work based on the concept, implementation, and evaluation of the current state of the RMF.

\subsection{Discussion of the results}

The ISO 27004 \cite{ISO_27004_2009} standard is very generalized. That generalization makes it possible to design and implement technical frameworks which are not covered as a standard or intended to be. A problem with the requirements and procedures of ISO 27004 are the individual organization based decisions. These decisions must be implemented individually and set before using the RMF, whereby conceptual decisions that are not organization-related could not be made. On the other hand, the separation also ensures that general non-organizational values are always implemented in the same way. The specific values can thus always be expected by the RMF in such a way that these values are also processed at the expected points. This process also ensures that the organization-unspecific values could always be disregarded in this thesis. \\ \\
Regarding the risk matrices, it is worth mentioning that the results of the measurement from the attacked model can be represented using the measurement from the original ML model. The problem here is that only minor and critical risks are considered. For the representation of the measurement results with a risk matrix values are missing, which are either fixed before or standardized. \\ \\
In the case of risk indicators and their values from the measurement methods could be measured alone according to the concept of this thesis. However, the evaluation in the measurement construct in the case study showed the problem that the values for attacker's effort could not be standardized. The attack time, computational resources and the steps to reflect the effort could not be summarized, so their values had to be considered individually. The problem of unification could therefore not be solved in this thesis, but nevertheless confirmed hypothesis \ref{itm:h4}. The individual values in comparison with values from another attack could show how high the effort is. Thus introduced arguments that related the effort with the probability of occurrence. \\ \\
The threat model used in this thesis could be included without any problems, with the exception of the omission of individual characteristics. The risk indicators used to calculate the extent of harm could be summed to one value, allowing research question \ref{itm:rq2} to be answered based on hypothesis \ref{itm:h1}.

\subsection{Proposals for possible future optimizations}

After implementing only backdoor attacks, it should be possible in the future to use attacks for different attack times. These attacks could increase the knowledge for vulnerabilities in the measured ML model. Further, the RMF could show how to prevent those attacks. The risk indicators for the attacker's effort should be unified to calculate a total risk value that represents the risk on the attacked ML model. Jianxing et al. \cite{DBLP:journals/access/JianxingHSH21} explain in their work that the level of risk could be suitably integrated on knowledge-based rules which this thesis did not do. This is because there is no knowledge-base which this thesis could use to integrate the risk value but could be implemented in the future. In order to increase the accuracy of the effort steps, it would be useful to add a weighting to the individual steps that describes how much effort is required for each step. \\
To find vulnerabilities that could execute a backdoor attack, \cite{DBLP:journals/corr/abs-1811-03728} describes an algorithm to identify these attacks based on activation clustering. This would improve the risk measurement and would help to improve the values of the risk indicators.
