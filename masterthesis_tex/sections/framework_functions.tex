% !TEX root = C:\Users\Jan\Documents\dev\Risk-Measurement-Framework\masterthesis_tex\masterthesis_main.tex
\section{Framework functions}
\label{sec:frame_func}

\begin{lstlisting}
  log(message, logging_levelname: str = 'INFO')
\end{lstlisting}

\noindent\textbf{message}
output in the log \\ \\
\textbf{logging\_levelname}
(Optional) string value to show the logging level \\ \\

The following functions are the attacks in the RMF: \\

\begin{lstlisting}
  art_poison_backdoor_attack(x, y, num_of_rand_images)
\end{lstlisting}

\noindent\textbf{x}
This argument takes the array of training images. \\ \\
\textbf{y}
This argument takes the array of labels that are assigned to the training images. \\ \\
\textbf{num\_of\_rand\_images}
This argument takes the number of untargeted images that should get a trigger pattern. \\ \\

\begin{lstlisting}
  clean_label(x, y)
\end{lstlisting}

\noindent\textbf{x}
This argument takes the array of training images. \\ \\
\textbf{y}
This argument takes the array of labels that are assigned to the training images. \\ \\

\begin{lstlisting}
  art_hidden_trigger_backdoor(x, y, target, source)
\end{lstlisting}

\noindent\textbf{x}
This argument takes the array of training images. \\ \\
\textbf{y}
This argument takes the array of labels that are assigned to the training images. \\ \\
\textbf{target}
This argument takes the target label. \\ \\
\textbf{source}
This argument takes the source labels that should be poisoned.

The following functions visualize the ML model training process: \\

\begin{lstlisting}
  dataset_visualization(class_num, train_number)
\end{lstlisting}

\noindent\textbf{class\_num}
This argument takes the number of labels. \\ \\

\textbf{train\_number}
This argument takes the training dataset to visualize it.

The following functions represent the risk indicators in the RMF: \\
These three functions form the computational resources risk indicator: \\
\begin{lstlisting}
  ram_resources()
\end{lstlisting}

\begin{lstlisting}
  cpu_resources()
\end{lstlisting}

\begin{lstlisting}
  gpu_resources()
\end{lstlisting}

\begin{lstlisting}
  accuracy_log(true_values, predictions, normalize=False)
\end{lstlisting}

\noindent\textbf{true\_values}
\\ \\
\textbf{predictions}
\\ \\
\textbf{normalize=False}
(Optional) the accuracy can be normalized but the default value is \textit{False} \\ \\

\begin{lstlisting}
  attackers_knowledge(attack)
\end{lstlisting}

\noindent\textbf{attack}
This argument takes the executed attack function to poison the training data \\

\begin{lstlisting}
  attackers_goal()
\end{lstlisting}

\begin{lstlisting}
  positive_negative_label(model, labels, predictions)
\end{lstlisting}

\noindent\textbf{model}
This argument takes the used classifier. \\ \\
\textbf{labels}
This argument takes the correct labels. \\ \\
\textbf{predictions}
This argument takes the predicted labels. \\ \\

\begin{lstlisting}
  attack_time(attack)
\end{lstlisting}

\noindent\textbf{attack}
This argument takes the executed attack function to poison the training data \\

\begin{lstlisting}
  attack_specificty(target)
\end{lstlisting}

\noindent\textbf{target}
This argument expects if the attack is targeted or untargeted \\

The following functions are used for the measurement construct: \\

\begin{lstlisting}
  mapping(low_l, high_l)
\end{lstlisting}

\noindent\textbf{low\_l}
This argument must be a list of all low-level attributes \\

\noindent\textbf{high\_l}
This argument must be a list of all high-level attributes \\

\begin{lstlisting}
  measurement_functions(base_measures)
\end{lstlisting}

\noindent\textbf{base\_measures}
This argument have to be a list of the assigend risk indicators \\ \\

\begin{lstlisting}
  analytical_model(base_mea_raw, derived_measures)
\end{lstlisting}

\noindent\textbf{base\_mea\_raw}
This argument takes a list of all base measures that are not used for the derived measures \\

\noindent\textbf{derived\_measures}
This argument takes a list of all derived measures from the measurement function \\

\begin{lstlisting}
  decision_criteria(*indicator)
\end{lstlisting}

\noindent\textbf{*indicator}
This \textbf{*args} is an argument that takes all indicators from the analytical model

The following function shows the data results of the risk measurement: \\

\begin{lstlisting}
  risk_matrix()
\end{lstlisting}

This function draws the risk matrix based on the return values from the \textit{decision\_criteria()} function.
