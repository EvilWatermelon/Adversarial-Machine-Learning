The following table explains and declares terms that are used in this thesis for standartization. The order of the table is organized by declaring a term before its relation to other terms.


% left fixed width:
\newcolumntype{L}[1]{>{\raggedright\arraybackslash}p{#1}}

% center fixed width:
\newcolumntype{C}[1]{>{\centering\arraybackslash}p{#1}}

% flush right fixed width:
\newcolumntype{R}[1]{>{\raggedleft\arraybackslash}p{#1}}


\begin{center}
  \begin{tabular}{ |L{4cm}|L{10cm}|  }
    \hline
    \rowcolor{lightgray} Term & Description \\ [0.5ex]
    International Organization for Standartization (ISO) & The ISO is an international statndard organization which declares international standards in all technical areas \cite{bsi_glossar_2021}. \\
    & \\
    \hline
    Information Security Management System (ISMS) & \\
    & \\
    \hline
    Model & A model in context of machine learning
    is a representation of a machine learning program
    that learned from the training sets \cite{google}. \\
    & \\
    \hline
    Prediction & Output of a model coming from an input \cite{google}. \\
    & \\
    \hline
    Machine Learning & Machine learning is a research field and also describes a program or a system that trains from
    input data a predictive model. That predictive model makes predictions
    from never-before-seen data \cite{google}. \\
    & \\
    \hline
    Threat & Security violation when an event could cause harm \cite{DBLP:journals/rfc/rfc4949}. \\
    & \\
    \hline
    Threat model & A threat model is a model to find security problems.
    This model should give a bigger picture away from the code \cite{shostack_2017}. \\
    & \\
    \hline
    Framework & A framework is a layered structure with a set of functions. It can specify programming interfaces, or offer programming tools \cite{FIPS1402}. \\
    & \\
    \hline
    Vulnerability & A vulnerability is an error that cause security
    relevant threats \cite{bsi_glossar_2021}. \\
    & \\
    \hline
    Attacker & A person that exploits potential system vulnerabilities \cite{FIPS1402}. \\
    & \\
    \hline
  \end{tabular}
\end{center}

\newpage

\begin{center}
  \begin{tabular}{ |L{4cm}|L{10cm}|  }
    \hline
    \rowcolor{lightgray} Term & Description \\ [0.5ex]
    Attacker's effort & \\
    & \\
    \hline
    Risk & Risk is the combination between the frequency of damage and
    the extent of damage. The damage is the difference between a
    planned and unplanned result \cite{bsi_glossar_2021}. \\
    & \\
    \hline
    Risk indicator & Risk indicators beeing used for assessing risks
    and predicting potential new risks \cite{Saluja2014RiskIF}. \\
    & \\
    \hline
    Property as a risk indicator & \\
    & \\
    \hline
    Proposals to measure risks & Proposals come from the thesis approaches and will be used to find risk indicators. \\
    & \\
    \hline
    Decision criteria & To describe the level of confidence in a given result, decision criteria pretend thresholds, targets, or patterns \cite{ISO_27004_2009}. \\
    & \\
    \hline
    Measure & Variable which is assigned as the measurement result \cite{ISO_27004_2009}. \\
    & \\
    \hline
    Analytical model & An algorithm that combines one or more derived measures with its associated decision criteria \cite{ISO_27004_2009}. \\
    & \\
    \hline
    Measurement function & An algorithm to combine two or more measures \cite{ISO_27004_2009}. \\
    & \\
    \hline
    Measurement method & Operations in a logical sequence \cite{ISO_27004_2009}. \\
    & \\
    \hline
    Measurement & Gathering information about the IT security system by using a measurement method, function, analytical model and decision criteria \cite{ISO_27004_2009}. \\
    & \\
    \hline
    Measurement result & One or more indicators with its interpretations which address a need for information \cite{ISO_27004_2009}. \\
    & \\
    \hline
    Information need & \\
    & \\
    \hline
  \end{tabular}
\end{center}

\begin{center}
  \begin{tabular}{ |L{4cm}|L{10cm}|  }
    \hline
    Base measure & \\
    & \\
    \hline
    Derived measure & \\
    & \\
    \hline
    Metric & In context of machine learning, a metric is a value that is
    optimized  in a machine learning program \cite{google}.
    In context of this thesis, metrics will be used often for
    measuring risks. \\
    & \\
    \hline
    High-level attributes & \\
    & \\
    \hline
    Low-level attributes & \\
    & \\
    \hline
  \end{tabular}
\end{center}
