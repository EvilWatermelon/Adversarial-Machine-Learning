The following table explains and declares terms that are used in this thesis for standartization. The order of the table is organized by declaring a term before its relation to other terms.


% left fixed width:
\newcolumntype{L}[1]{>{\raggedright\arraybackslash}p{#1}}

% center fixed width:
\newcolumntype{C}[1]{>{\centering\arraybackslash}p{#1}}

% flush right fixed width:
\newcolumntype{R}[1]{>{\raggedleft\arraybackslash}p{#1}}


\begin{center}
  \begin{tabular}{ |L{4cm}|L{10cm}|  }
    \hline
    \rowcolor{lightgray} Term & Description \\ [0.5ex]
    International Organization for Standartization (ISO) & The ISO is an international statndard organization which declares international standards in all technical areas \cite{bsi_glossar_2021}. The ISO 27004 will be the basis for designing and implementing the Risk-Measurement-Framework. \\
    & \\
    \hline
    Information Security Management System (ISMS) & \\
    & \\
    \hline
    Model & A model in context of machine learning
    is a representation of a machine learning system
    that learned from the training sets \cite{google}. \\
    & \\
    \hline
    Prediction & A predicition is an output of a machine learning model coming from an input e.g. data that the machine learning has never seen before \cite{google}. \\
    & \\
    \hline
    Machine Learning & Machine learning is a research field and also describes a program or a system that trains from
    input data a predictive model. That predictive model makes predictions
    from never-before-seen data \cite{google}. Machine learning models will be measured for security risks in this thesis. \\
    & \\
    \hline
    Threat & A threat is a security violation when an event could cause harm \cite{DBLP:journals/rfc/rfc4949}. Threats will be find in the threat models and the risk measurement. \\
    & \\
    \hline
    Threat model & A threat model is a model to find security problems.
    This model should give a bigger picture away from the code \cite{shostack_2017}. A threat model will be used to declare different characeristics with their relation between attacks and an attacker. \\
    & \\
    \hline
    Framework & A framework is a layered structure with a set of functions. It can specify programming interfaces, or offer programming tools \cite{FIPS1402}. In this thesis a framework will be designed, implemented and evaluated with a case study. \\
    & \\
    \hline
    Vulnerability & A vulnerability is weakness of a program or system which could be exploited by a threat. \cite{FIPS1402}. The vulnerability term will be in this thesis used in context with the attacks and vulnerabilities in ML models. \\
    & \\
    \hline
    Attacker & A person that exploits potential system vulnerabilities \cite{FIPS1402}. For this thesis an attacker stands in relation with its effort to attack an machine learning model. \\
    & \\
    \hline
  \end{tabular}
\end{center}

\newpage

\begin{center}
  \begin{tabular}{ |L{4cm}|L{10cm}|  }
    \hline
    \rowcolor{lightgray} Term & Description \\ [0.5ex]
    Attacker's effort & The attacker's effort is a term that represent everything that an attacker do to attack the machine learning model. In this thesis the attacker's effort represent probability of occurrence, among other things. \\
    & \\
    \hline
    Risk & Risk is the combination between the frequency of damage and
    the extent of damage. The damage is the difference between a
    planned and unplanned result \cite{bsi_glossar_2021}. The risk calculation will be used as the final result after the risk measurement. \\
    & \\
    \hline
    Risk indicator & Risk indicator is the generic term for all indications that contribute to the risk measurement in the framework, such as properties, values, and metrics. They will be used to measure risks and to represent the results as data, logs, or visual representations of the risk measurement. \\
    & \\
    \hline
    Decision criteria & To describe the level of confidence in a given result, decision criteria pretend thresholds, targets, or patterns \cite{ISO_27004_2009}. \\
    & \\
    \hline
    Measure & Variable which is assigned as the measurement result \cite{ISO_27004_2009}. \\
    & \\
    \hline
    Analytical model & An algorithm that combines one or more derived measures with its associated decision criteria \cite{ISO_27004_2009}. \\
    & \\
    \hline
    Measurement function & An algorithm to combine two or more measures \cite{ISO_27004_2009}. \\
    & \\
    \hline
    Measurement method & Operations in a logical sequence \cite{ISO_27004_2009}. \\
    & \\
    \hline
    Measurement & Gathering information about the IT security system by using a measurement method, function, analytical model and decision criteria \cite{ISO_27004_2009}. \\
    & \\
    \hline
    Measurement result & One or more indicators with its interpretations which address an information need \cite{ISO_27004_2009}. \\
    & \\
    \hline
    Information need & \\
    & \\
    \hline
  \end{tabular}
\end{center}

\begin{center}
  \begin{tabular}{ |L{4cm}|L{10cm}|  }
    \hline
    Base measure & \\
    & \\
    \hline
    Derived measure & \\
    & \\
    \hline
    Metric & In context of this thesis, metrics will be used often to represent different results (i.e. accuracy) of the machine learning model. \\
    & \\
    \hline
    High-level attributes & High-level attributes are subjective attributes that represent all characeristics of an attacker's effort. This term will be used in context of a formal threat model. \\
    & \\
    \hline
    Low-level attributes & Low-level attributes are objective attributes that represent all characeristics of the attack's data. This term will be used in context of a formal threat model. \\
    & \\
    \hline
  \end{tabular}
\end{center}

This thesis uses the terms decision criteria, measure, analytical model, measurement function, measurement method, measurement, information need, base measure, and derived measure in context with the ISO 27004 standard to design and implement the Risk-Measurement-Framework.
