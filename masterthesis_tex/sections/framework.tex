\section{The conceptual framework}
\label{sec:conFrame}

In contrast to Schwerdtner et al., the framework of this thesis concentrates on training, especially Risk Measurement before and during training of the ML model.
The conceptual framework discusses and explains the RMF. The RMF is a conceptual and technical framework which measures risks of backdoor attacks and measures the attacker effort. The attacker effort
is measured by objective properties. These objective properties are the base of the risk indicators for the attacker effort explained in the following subsection. Objective properties

\subsection{UML diagrams}

\subsection{Finding the attacker's effort}

\subsubsection*{Using threat models to find risk indicators to measure the attackers effort}

\subsection{Characteristics of backdoor attacks}

\subsection{Risk indicators}
\label{sec:risk_indicators}

The RMF measure risks by so called risk indicators. Properties, attributes and proposals are the basis for the risk indicators, among other things. Breier et al. in subsection \ref{sec:approaches} present proposals that are the approach for the proposals for the risk indicators.
