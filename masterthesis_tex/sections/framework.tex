\section{The conceptual framework}
\label{sec:conFrame}

In contrast to Schwerdtner et al., the framework of this thesis concentrates on training, especially Risk Measurement before and during training of the ML model.
The conceptual framework discusses and explains the RMF. The RMF is a conceptual and technical framework which measures risks of backdoor attacks and measures the attackers effort. The attackers effort is measured by high-level und low-level properties. Section \ref{sec:relWork}

\subsection{Finding the attacker's effort}



\subsubsection*{Using threat models to find risk indicators to measure the attackers effort}

\subsection{Characteristics of backdoor attacks}

\subsection{Risk indicators}
\label{sec:risk_indicators}

The RMF measure risks by so called risk indicators. Properties, threat models and proposals are the basis for the risk indicators. Breier et al. in subsection \ref{sec:approaches} present
proposals that are the approach for the proposals of the risk indicators. Doynikova et al. presents a formal threat model to find the attackers effort. 
