\section{The conceptual framework}
\label{sec:conFrame}

In contrast to Schwerdtner et al., the framework of this thesis concentrates on training, especially risk measurement before and during training of the ML model.
The conceptual framework discusses and explains the design of the RMF. The RMF is a technical framework which measures risks of backdoor attacks and measures the attackers effort.

\subsection{Characteristics of backdoor attacks}


\subsection{Types of backdoor attacks}

The following backdoor attacks should represent what they can achieve when using them. Further, this subsection should show the basis of the backdoor attacks that are used in the RMF.

\subsection{Measure risks based on the standards}


\subsubsection*{Derivate the standards for machine learning}


\subsubsection*{The theory behind the ART backdoor attacks}

\textit{PoisoningAttackBackdoor} and \textit{PoisoningAttackCleanLabelBackdoor} are the two backdoor attacks in the framework. Gu et al. \cite{DBLP:journals/corr/abs-1708-06733} explain \textit{PoisoningAttackBackdoor} attacks. The goal of this backdoor attack is to change their labels to a target label. This happens by attacking a random small selection of the training set and apply a backdoor trigger into the inputs \cite{turner2018clean}. Gu et al. show in their work different backdoor attacks and do a case study with a traffic sign detection attack. In their work, Gu et al. developed a neural network with a backdoor trigger. The evaluated backdoors are a single pixel backdoor and a pattern backdoor. The single pixel backdoor increase the brightness of a pixel and the pattern backdoor adds a pattern of bright pixels in an image. The implemented attacks from Gu et al. are Single Target attack and an All-to-All attack. Single Target attack use the single pixel backdoor by changing a label from a digit $i$ as a digit $j$. Gu et al. explained that the test data are not available for the attacker. The error rate for their Convolutional Neural Network (CNN) is 0.05\%. The error rate with the backdoored images increases at most to 0.09\%. An All-to-All attack change a digit label $i$ to $i + 1$. After testing the All-to-All attack the originial ML have a error rate of 0.03\% while the ML with the backdoored image have an average error of 0.56\%. \\
In their work, Turner et al. \cite{turner2018clean} explain
\textit{PoisoningAttackCleanLabelBackdoor} attacks. Turner et al. show an approach for executing backdoor attacks by utilizing adversarial examples and GAN-generated data. The point where Turner et al. start is analyzing effectiveness of Gu et al. attack while a simple technique is applied for data filtering. Turner et al. discovered that the poisoned inputs are outliers and are clearly wrong from the human inspection side. The attack would be ineffective if its rely solely on poisoned inputs which are labeled correctly and evade such filtering. At this point Turner et al. created an approach that do poisoned inputs which appear plausible to humans. The inputs need small changes to make them harder while classify them but the original label must still remain plausible. This transformation is performed by a GAN-based interpolation and adversarial bounded pertubations. GAN-based interpolation takes each input into the GAN latent space \cite{DBLP:conf/nips/GoodfellowPMXWOCB14} and then interpolate poisoned samples to an incorrect class. Adversarial bounded pertubations uses a maximization method to maximize the loss of the pre-trained ML model on poisoned inputs while staying around the original input.

\subsubsection*{Additional backdoor attacks to increase the possible extent of damage}


\subsection{Finding the attacker's effort}

Subsection \ref{sec:threat} explained a formal threat model to find the attackers effort with high-level and low-level properties where the low-level properties are mapped to with the high-level properties. At first this subsection will discuss which of the characteristics are useful to find the attackers effort for attacking a ML model. Regarding to the mapping between the properties, the low-level properties will be discussed at first.

\subsection{Using the formal threat model}


\subsubsection*{The low-level properties}


\subsubsection*{The high-level properties}

\subsubsection*{Mapping the low-level with the high-level properties}


\subsubsection*{Derivate the properties to machine learning}


\subsection{Risk indicators}
\label{sec:risk_indicators}

The RMF measure risks by so called risk indicators. Properties, threat models and proposals are the basis for the risk indicators. Breier et al. in subsection \ref{sec:approaches} present
proposals that are the approach for the proposals of the risk indicators. Doynikova et al. presents a formal threat model to find the attackers effort.

\subsubsection*{Properties, proposals and characteristics derived from classic IT security}


\subsubsection*{Correlations of the properties, proposals and characteristics}


\subsection{Evaluation methods for the measured risks}


\subsubsection*{Analyze the dataset for vulnerabilites}


\subsubsection*{Logging the execution of the attack}


\subsubsection*{Machine learning metrics for risk measurement}


\subsubsection*{Python plots}


\subsubsection*{Calculate the risks}
