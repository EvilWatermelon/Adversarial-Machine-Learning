\section{Implementation}
\label{sec:implementation}

The technical RMF uses Python 3.7 as the programming language and ART as the basis. Beside the attacks given by the ART, there is a function from the technical RMF to execute individual attacks. This technical RMF should be used a step ahead of using the framework of Schwerdtner et al.

\subsection{Structure of the RMF}

\subsubsection*{Directory tree}

The RMF is structured as follows:

\dirtree{%
.1 rmf/.
.2 attacks/.
.3 art/.
.4 backdoors.py.
.2 metrics/.
.3 log.py.
.2 visualizations/.
.3 plot.py.
.2 log\_file.log.
.2 case\_study.py.
}

\subsection{Using ART as the basis for the technical framework}

The ART implemented two backdoor attacks which will be explained in \ref{sec:backdoors_art}. Since art is an open-source technical framework, the two backdoor attacks can also be used as a basis for simplifying the implementation of other attacks.

\subsection{Implementing backdoor attacks}

\subsubsection*{Backdoor attacks from the ART}
\label{sec:backdoors_art}

\subsubsection*{Additional attacks for the RMF}

Beside the attacks that are called from functions of ART it must be possible to implement and execute new attacks for the evaluation to measure the attackers knowledge, skills and extent
of damage.

\subsection{Build in the risk indicatiors}

The risk indicators are the main part for the risk measurement.

\subsection{Measuring risks with the risk indicators}

\subsection{Implementation of the logging function}

Show measured risks is able with logging from the Python logging module. The function waits for two parameters. A message string and the wanted logging level (i.e. INFO or DEBUG). The called log function in the RMF could look like this:
\begin{lstlisting}
  log(f"{variable_name}", 'INFO')
\end{lstlisting}

In order not to depend on the different ML libraries the rmf gets its own functions of the different metrics. That increases the support of different Python libraries for ML risk
measurement. The accuracy of the predicitions are calculated as follows: \\
\begin{center}
  $Accuracy = \frac{True Positives + True Negatives}{True Positives + True Negatives + False Positives + False Negatives}$
\end{center}

\subsection{Implementation of the visualization}

For the visualization Python modules like sci-kit learn have implemented different plots that are signed as metrics.
